\documentclass[french,a4paper]{article}
\setcounter{tocdepth}{4}
\setcounter{secnumdepth}{4}
\usepackage{booktabs}
\newcommand{\tabitem}{\textbullet~~}\title{Compte rendu de réunion}
\usepackage[bottom=2.5cm,top=2.5cm,left=2.5cm,right=2.5cm]{geometry}
\author{Noé Steiner - Alexis Marcel - Lucas Laurent - Mathias Aurand-Augier}
\date{3 Mai 2023}
\begin{document}
\maketitle

\section*{\underline{Projet PPII - Compte rendu n°3 - réunion technique}}

\begin{table}[!htb]
  \centering
  \begin{tabular}{| p{7cm} | p{7cm} |}
    \hline
    \multicolumn{1}{|c|}{ Participants:} & \multicolumn{1}{c|}{Lieu:} \\
    \hline
    \tabitem Alexis : Présent\newline
    \tabitem Noé : Présent\newline
    \tabitem Lucas : Absent\newline
    \tabitem Mathias : Présent                      &
    \tabitem Le 3 mai 2023\newline
    \tabitem De 15h à 17h\newline
    \tabitem Visioconférence sur Discord                                         \\
    \hline
  \end{tabular}
\end{table}

\subsection*{\textit{Ordre du jour:}}

\begin{itemize}
  \item Cloture de la partie 1
  \item Établir un plan pour la deuxième étape du projet
  \item Discussion sur la logique de programmation pour la simulation de plusieurs utilisateurs et la gestion des files d'attente
  \item Etudes des différentres structures de données possibles
\end{itemize}

\subsubsection*{\textit{Information échangées}}
\begin{itemize}
  \item La deuxième version de l'algo (optimisé) est fonctionnelle.
  \item 
\end{itemize}
\subsubsection*{\textit{Remarques / Questions}}
Aucune question ou remarque spécifique relevée.

\subsection*{\textit{Décisions}}
\begin{itemize}
  \item La structure choisi pour la simulation est 
\end{itemize}

\subsection*{\textit{Actions à suivre / Todo list}}
\begin{itemize}
  \item 
\end{itemize}

\subsection*{\textit{Date de la prochaine réunion}}
La prochaine réunion aura lieu le lundi 24 Avril 2023 (horaire à définir). 
En attendant, de multiples séances de travail seront organisés pour travailler ensemble sur le code.

\end{document}
