\documentclass[french,a4paper]{article}
\setcounter{tocdepth}{4}
\setcounter{secnumdepth}{4}
\usepackage{booktabs}
\newcommand{\tabitem}{\textbullet~~}\title{Compte rendu de réunion}
\usepackage[bottom=2.5cm,top=2.5cm,left=2.5cm,right=2.5cm]{geometry}
\author{Noé Steiner - Alexis Marcel - Lucas Laurent - Mathias Aurand-Augier}
\date{5 Avril 2023}
\begin{document}
\maketitle

\section*{\underline{Projet PPII - Compte rendu n°01 - réunion de lancement}}

\begin{table}[!htb]
  \centering
  \begin{tabular}{| p{7cm} | p{7cm} |}
    \hline
    \multicolumn{1}{|c|}{ Participants:} & \multicolumn{1}{c|}{Lieu:} \\
    \hline
    \tabitem Alexis : Présent\newline
    \tabitem Noé : Présent\newline
    \tabitem Lucas : Présent\newline
    \tabitem Mathias : Présent                      &
    \tabitem Le 5 avril 2023\newline
    \tabitem De 10h à 12h\newline
    \tabitem Visioconférence sur Discord                                         \\
    \hline
  \end{tabular}
\end{table}

\subsection*{\textit{Ordre du jour:}}

\begin{itemize}
  \item Présentation du projet et des objectifs
  \item Comment modéliser le problème ?
  \item Etat de l'art : quels sont les algos existants ?
  \item Choix de l'algorithme utilisé
  \item Répartition des tâches
  \item Établissement du calendrier prévisionnel
\end{itemize}


\subsubsection*{\textit{Information échangées}}
\begin{itemize}
  \item Vue globale sur le projet : ce que nous allons faire et comment nous allons le faire.
  \item Choix de la structure de donnée : arbre ou graphe ?
  \item Choix de l'algorithme de parcours minimal : Dijkstra ou A* ?
  \item Comment s'organiser pour écrire sur le même document en même temps ?
\end{itemize}
\subsubsection*{\textit{Remarques / Questions}}
Aucune question ou remarque spécifique relevée.

\subsection*{\textit{Décisions}}
\begin{itemize}
  \item La structure de donnée sera un graphe.
  \item L'algorithme de parcours minimal sera Dijkstra.
\end{itemize}

\subsection*{\textit{Actions à suivre / Todo list}}
\begin{itemize}
  \item Mettre en place la structure de graphe : avec les fonctions classique de structure de données (création, destruction, recherche...) (Tout le monde)
  \item Faire un premier patron de l'algorithme de Dijkstra à partir du cours de MSED (Tout le monde)
  \item Commencer l'implémentation de l'algorithme (Tout le monde)
\end{itemize}

\subsection*{\textit{Date de la prochaine réunion}}
La prochaine réunion aura lieu le Mercredi 19 Avril 2023 (horaire à définir). 
En attendant, de multiples séances de travail seront organisés pour travailler ensemble sur le code.

\end{document}

\end{document}
